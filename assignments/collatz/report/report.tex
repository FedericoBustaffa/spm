\documentclass[12pt, a4paper]{article}

\usepackage[T1]{fontenc}
\usepackage[english]{babel}
\usepackage{mathtools, amsmath, amssymb, amsthm}
\usepackage[hidelinks]{hyperref}
\usepackage{tabularx}
\usepackage{svg}
\usepackage{caption}
\usepackage{float}
\usepackage{booktabs}

\usepackage{minted}
\definecolor{minted_bg}{rgb}{0.9, 0.9, 0.9}
\usemintedstyle{colorful}

\setminted[cpp]{
	tabsize=4,
	% linenos=true,
	bgcolor=minted_bg,
	fontsize=\small,
	mathescape=true
}

\title{Assignment 2\\Collatz Steps}
\author{Federico Bustaffa}
\date{12/03/2025}

\begin{document}

\maketitle
\tableofcontents
\clearpage

\section{Introduction}

The aim of the assignment was to implement, through C++ threads two parallel
versions of an algorithm the computes the \verb|collatz_steps| function on each
value of one or more given ranges.

\begin{minted}{cpp}
  uint64_t collatz_steps(uint64_t n) {
    uint64_t steps = 0;
    while (n != 1) {
      n = (n % 2 == 0) ? n / 2 : 3 * n + 1;
      steps++;
    }

    return steps;
  }
\end{minted}

One implementation has to dispatch the workload through a static policy, in
particular a \emph{block-cyclic} policy, while the other method implement a
dynamic policy like a shared threadpool queue.

In the end a theoretical analysis is performed using the \emph{work-span} model
to estimate the thoretical speedup for both implementations.

\end{document}
